\documentclass[12]{beamer}
\usepackage{multicol}
\usepackage{enumitem}
\usepackage[obeyspaces]{url}
\usepackage{listings}
\lstnewenvironment{python}{\lstset{language=python}}{}
\lstnewenvironment{java}{\lstset{style=customjava,language=java}}{}
\lstnewenvironment{javaxs}{\lstset{basicstyle=\tiny,style=customjava,language=java}}{}

\lstdefinestyle{customjava}{
  belowcaptionskip=1\baselineskip,
  breaklines=true,
  language=java,
  showstringspaces=false,
  basicstyle=\footnotesize\ttfamily,
  keywordstyle=\bfseries\color{green!40!black},
  commentstyle=\itshape\color{purple!40!black},
  identifierstyle=\color{blue},
  stringstyle=\color{orange},
}

\usetheme[progressbar=frametitle]{metropolis}
\setbeamertemplate{frame numbering}[fraction]
\usefonttheme{metropolis}
\usecolortheme{spruce}
\setbeamercolor{background canvas}{bg=white}

\definecolor{mygreen}{rgb}{125, 5, 25}
\usecolortheme[named=mygreen]{structure}



\title{Week 10 Tutorial}
\author{Brae}
\institute {CSSE2002: Programming in the Large}

\setbeamercovered{transparent=5}


\begin{document}
\metroset{block=fill}

\begin{frame}
\titlepage
\end{frame}

\begin{frame}[t, fragile]{Question One} \vspace{4pt}

\begin{java}
public void f(int arr[]) {
    int total=0;
    for (int i=0;i<arr.length;++i) {
        total+=arr[i];
        i++;
    }
    System.out.println(total);
}
\end{java}
What would be output to the terminal if f was called with the array \{1,2,3,3,5,6,7\}?

\begin{onlyenv}<2>
\textbf{Note:} Terminal means screen/console/output window
\end{onlyenv}

\begin{onlyenv}<3>
Note that \textbf{i} is incremented in both the afterthought and loop body.

$1 + 3 + 5 + 7 = 16$
\end{onlyenv}

\end{frame}

\begin{frame}[t, fragile]{Question Two} \vspace{4pt}

\begin{java}
public int g(int v) {
    if (v>5) {
        return 0;
    }
    if (v<=0) {
        return Math.abs(v)+g(v+2);
    }
    return (v-1)+g(v+1);
}
\end{java}

What would be returned by the following calls? g(2), g(0), g(-5)

\begin{onlyenv}<2->
\textbf{Recursive Desk Check} - Can you do it?
\end{onlyenv}

\begin{onlyenv}<3->
\textbf{Hint:} Sometimes you can use your other working to come to a solution quicker
\end{onlyenv}

\end{frame}

\begin{frame}[t, fragile]{Question Two Answers} \vspace{4pt}
\begin{center}
\begin{onlyenv}<1>
\begin{tabular}{ |c|c|c| } 
\hline
g(v) & returned & evaluated returned \\
 \hline
 g(2) & 1 + g(3) & 1 + 9 = 10 \\ 
 g(3) & 2 + g(4) & 2 + 7 = 9 \\ 
 g(4) & 3 + g(5) & 3 + 4 = 7 \\ 
 g(5) & 4 + g(6) & 4 + 0 = 4 \\ 
 g(6) & 0 & 0 \\
 \hline
\end{tabular}
\end{onlyenv}
\begin{onlyenv}<2>
\begin{tabular}{ |c|c|c| } 
\hline
g(v) & returned & evaluated returned \\
 \hline
 g(2) & 1 + g(3) & 1 + 9 = 10 \\ 
 g(3) & 2 + g(4) & 2 + 7 = 9 \\ 
 g(4) & 3 + g(5) & 3 + 4 = 7 \\ 
 g(5) & 4 + g(6) & 4 + 0 = 4 \\ 
 g(6) & 0 & 0 \\ 
 \hline
 g(0) & 0 + g(2) & 0 + 10 = 10 \\
 \hline
\end{tabular}
\end{onlyenv}
\begin{onlyenv}<3>
\begin{tabular}{ |c|c|c| } 
\hline
g(v) & returned & evaluated returned \\
 \hline
 g(2) & 1 + g(3) & 1 + 9 = 10 \\ 
 g(3) & 2 + g(4) & 2 + 7 = 9 \\ 
 g(4) & 3 + g(5) & 3 + 4 = 7 \\ 
 g(5) & 4 + g(6) & 4 + 0 = 4 \\ 
 g(6) & 0 & 0 \\ 
 \hline
 g(0) & 0 + g(2) & 0 + 10 = 10 \\
 \hline
 g(-5) & 5 + g(-3) & 5 + 14 = 19 \\
 g(-3) & 3 + g(-1) & 3 + 11 = 14 \\
 g(-1) & 1 + g(1) & 1 + 10 = 11 \\
 g(1) & 0 + g(2) & 0 + 10 = 10 \\
 \hline
\end{tabular}
\end{onlyenv}
\end{center}
\end{frame}

\begin{frame}[t, fragile]{Question Three} \vspace{4pt}

\begin{onlyenv}<1>
\begin{java}
public class V {
    ...
    public static double f(int v);
    ...
}
\end{java}

Write a single JUnit4 test method which checks that:
\begin{itemize}
\item[$\bullet$] f(2) returns a value greater than 0.
\item[$\bullet$] f(0) returns 0.5
\item[$\bullet$] f(-1) throws a NullPointerException
\end{itemize}
\end{onlyenv}

\begin{onlyenv}<2>
\textbf{Exam Tips}\\

\begin{itemize}

\item[$\bullet$] You can always assume any common imports have been imported.

\item[$\bullet$] If you are asked to write code fragments, you can assume the class and containing methods are defined.

\item[$\bullet$] If you are asked to write a method, write a whole method.

\item[$\bullet$] If you are asked to write a program, write the whole class including a main method.

\item[$\bullet$] If you believe that you are using types which are ambiguous, put a comment.

\end{itemize}
\end{onlyenv}

\begin{onlyenv}<3->
\begin{java}
@Test
public void test() {
    Assert.assertTrue(V.f(2)>0); // remember f is static
    Assert.assertEquals(0.5, V.f(0), 0.001);
    try {
        f(-1);
        Assert.fail();
    } catch (NullPointerException ex) {
        
    }
}
\end{java}

\end{onlyenv}
\begin{onlyenv}<4>
Possible Assumptions:
\begin{itemize}
\item static import of f from V i.e. \texttt{import static V.f;}
\item static import of Assert.* i.e. \texttt{import static Assert.*;}
\end{itemize}
\end{onlyenv}

\begin{onlyenv}<5>
Notes:
\begin{itemize}
\item assertEquals(0.5, V.f(0)) is not correct. 
\item Catching a broader exception than you need also not correct.
\end{itemize}
\end{onlyenv}

\end{frame}

\begin{frame}[t, fragile]{Question Four} \vspace{4pt}

\begin{onlyenv}<1>
%\begin{javaxs}
%public class Bag {
%    /**
%    * Create an empty Bag
%    */
%    public Bag();
%    /**
%    * @param num number to search for
%    * @return number of times num appears in the Bag (0 if not present).
%    */
%    public int getCount(int num);
%    /**
%    * @param num number to add to Bag
%    */
%    public void add(int num);
%    /**
%    * @param num number to remove (If present multiple times, only remove one instance)
%    */
%    public void remove(int num);
%    /**
%    * @return a String containing comma separated value:times pairs (where ‘‘times’’ is %the number of times the value occurs in the Bag.
%    * for example: "1:1,5:1,3:7"
%    */
%    @Override
%    public String toString();
%}
%\end{javaxs}
\end{onlyenv}


\begin{onlyenv}<2>

A Bag is an unordered collection of ints where each value can appear multiple times.
For example a bag could contain 1, 5 and seven 3s.
\begin{itemize}
\item[A)] What member variables would you need to add to the class?
\item[B)] Implement the constructor.
\item[C)] Implement add
\item[D)] Implement remove
\item[E)] Implement getCount
\end{itemize}
\end{onlyenv}

\begin{onlyenv}<3->
\begin{itemize}
\item[A)] List\onslide<4->{: can add and remove items from the end of the list and loop through counting occurences for getCount}

\item[B)] Set\onslide<5->{: seems like a good idea but the items in the bag can appear multiple times}

\item[C)] Map\onslide<6->{: map values to number of times they appear. Needs filtering to not report on things that were added and then removed}
\end{itemize}
\end{onlyenv}

\end{frame}

\end{document}
