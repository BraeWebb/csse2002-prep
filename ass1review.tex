\documentclass[12]{beamer}
\usepackage{multicol}
\usepackage{enumitem}
\usepackage[obeyspaces]{url}
\usepackage{listings}
\lstnewenvironment{python}{\lstset{language=python}}{}
\lstnewenvironment{java}{\lstset{style=customjava,language=java}}{}

\lstdefinestyle{customjava}{
  belowcaptionskip=1\baselineskip,
  breaklines=true,
  language=java,
  showstringspaces=false,
  basicstyle=\footnotesize\ttfamily,
  keywordstyle=\bfseries\color{green!40!black},
  commentstyle=\itshape\color{purple!40!black},
  identifierstyle=\color{blue},
  stringstyle=\color{orange},
}

\usetheme[progressbar=frametitle]{metropolis}
\setbeamertemplate{frame numbering}[fraction]
\usefonttheme{metropolis}
\usecolortheme{spruce}
\setbeamercolor{background canvas}{bg=white}

\definecolor{mygreen}{rgb}{125, 5, 25}
\usecolortheme[named=mygreen]{structure}



\title{Assignment One Review}
\author{Brae}
%\institute{ajou university.}
\institute {CSSE2002: Programming in the Large}

\setbeamercovered{transparent=5}


\begin{document}
\metroset{block=fill}
\begin{frame}
\titlepage
\end{frame}

\begin{frame}[t, fragile]{Common Mistakes - Abstract Thing} \vspace{4pt}
The JavaDocs specified that the \texttt{Thing} class was an abstract class.

\begin{java}
public class Thing {
	...
}

public abstract class Thing {
	...
}
\end{java}
\end{frame}

\begin{frame}[t, fragile]{Common Mistakes - Code Duplication} \vspace{4pt}
Code duplication for replacing strings.

\begin{onlyenv}<2>
\textbf{Don't do this}
\begin{java}
public Thing(String shortDescription, String longDescription) {
    this.shortDescription = shortDescription.replace('\n', '*').replace(';', '*').replace('\r', '*');
    this.longDescription = longDescription.replace('\n', '*').replace(';', '*').replace('\r', '*');
}

protected void setShort(String shortDescription) {
    this.shortDescription = shortDescription.replace('\n', '*').replace(';', '*').replace('\r', '*');
}
\end{java}
\end{onlyenv}

\begin{onlyenv}<3>
Abstract away repeated code
\begin{java}
public Thing(String shortDescription, String longDescription) {
    this.shortDescription = replaceDescription(shortDescription);
    this.longDescription = replaceDescription(longDescription);
}

private String replaceDescrition(String description) {
    return description.replace('\n', '*')
            .replace(';', '*')
            .replace('\r', '*');
}
\end{java}
\end{onlyenv}

\end{frame}

\begin{frame}[t, fragile]{Common Mistakes - Member Duplication} \vspace{4pt}

\begin{onlyenv}<1>
\begin{java}
public class Explorer extends Thing implements Mob {

    private String shortDescription;
    private String longDescription;

    public Explorer(String shortDescription, String longDescription, int health) {
        super(shortDescription, longDescription);
        this.shortDescription = shortDescription;
        this.longDescription = longDescription;
        ...
    }
}
\end{java}
\end{onlyenv}

\begin{onlyenv}<2>
\begin{java}
public class Explorer extends Thing implements Mob {

    public Explorer(String shortDescription, String longDescription, int health) {
        super(shortDescription, longDescription);
        ...
    }
}
\end{java}
\end{onlyenv}

\end{frame}

\begin{frame}[t, fragile]{Common Mistakes - String Comparison} \vspace{4pt}

Strings need to be compared using the .equals method not the comparison operator.

\begin{java}
String first = "hello";
String second = "hello";

if (first == second) {} // wrong
if (first.equals(second)) {} // right
\end{java}

\end{frame}

\begin{frame}[t, fragile]{Common Mistakes - Style} \vspace{4pt}
Horizontal space is required on both sides of any binary or ternary operator.

Separate any reserved word, such as if, for or catch, from an open parenthesis (() that follows it on that line.

\begin{java}
if (x) {} // right
if(x){} //wrong
if (x){} //wrong
if(x) {} //wrong
\end{java}

\end{frame}

\begin{frame}[t, fragile]{Common Mistakes - Style} \vspace{4pt}

\begin{block}{Name Case}
\textbf{Variable names} should be in camelCase\\
\textbf{Class names} should be in PascalCase\\
\textbf{Constants} should be in \verb|SCREAMING_SNAKE_CASE|\\
\textbf{Method names} should be in camelCase\\[10pt]
\end{block}

\begin{onlyenv}<2->
\begin{java}
public class MyClassName {
	private static final int MAX_SCORE = 1000;
	private int bestScore = 0;

	public int getBestScore() {
		int myScore = 1;
		return myScore;
	}
}
\end{java}
\end{onlyenv}
\end{frame}

\begin{frame}[t, fragile]{Common Mistakes - Style} \vspace{4pt}

There is no line break after a curly brace if it is followed
by else or a comma.

\textbf{Wrong}
\begin{java}
if (x) {
  
}
else {
  
}
\end{java}

\textbf{Right}
\begin{java}
if (x) {
  
} else {
  
}
\end{java}

\end{frame}

\begin{frame}[t, fragile]{Common Mistakes - Style} \vspace{4pt}

One line if statements

\begin{java}
if (x) { return y; }
\end{java}

Braces are used with if, else, for, do and while statements,
even when the body is empty or contains only a single statement.

\begin{java}
if (x) return y;
\end{java}

\end{frame}

\begin{frame}[t, fragile]{Common Mistakes - Commenting} \vspace{4pt}

\textbf{Class Comments}
Used at the top of every file. Explains important details about the class.

\begin{java}
/*
 * An exception which is thrown when the programmer becomes unhappy
 */
 public class UnhappyProgrammerException extends Exception {

 }
\end{java}

\textbf{Inline Comments}
Used when code is not immediantly obvious.

\begin{java}
// subtracts one from all elements in the array
for (int i = 0; i < numbers.length; i++) {
    numbers[i] = numbers[i] - 1;
}
\end{java}

\end{frame}

\begin{frame}[t, fragile]{Common Mistakes - Commenting} \vspace{4pt}

\textbf{Method Comments}
Used to explain the purpose of a method. In assignment two you will
be using JavaDcos comments for methods.

\end{frame}

\end{document}
