\documentclass[week2]{csse2002}

\usepackage{slides}
\author{Brae Webb}

\title{CSSE2002 Week 2 Tutorial}
 
\begin{document}

\begin{frame}
\maketitle
\end{frame}

\begin{topic}{Who are you?}
\begin{itemize}
\item Degree?
\item Masters?
\item Year?
\end{itemize}
\end{topic}

\begin{topic}{Who am I?}
\begin{itemize}
\item I'm Brae!
\item I've been a CSSE1001 for a while now
\item Computer Science Honours
\end{itemize}
\end{topic}

\begin{topic}{Welcome to CSSE2002}
What should I expect from this course?

\begin{itemize}
    \begin{subtopic}{2-}
    \item To learn the basics of java
    \end{subtopic}
    \begin{subtopic}{3-}
    \item To learn some ways to ease complexity of large projects, including:
    \end{subtopic}
    \begin{itemize}
        \begin{subtopic}{3-}
        \item documentation and program specifications
        \end{subtopic}
        \begin{subtopic}{4-}
        \item inheritance and programming patterns
        \end{subtopic}
        \begin{subtopic}{5-}
        \item debugging programs
        \end{subtopic}
        \begin{subtopic}{6-}
        \item testing your programs to ensure they are correct
        \end{subtopic}
    \end{itemize}
    \begin{subtopic}{7-}
    \item Three interconnected assignments coded in java
    \end{subtopic}
\end{itemize}

\begin{subtopic}{3}
Large projects get complex fast!

This course will teach some techniques for reducing complexity such as documentation and program specifications.

Documentation and program specification minimises the amount of code any one programmer needs to understand and remember.
\end{subtopic}

\begin{subtopic}{5}
How do you find bugs in your programs?

\begin{python}
print("name variable contains", name)
\end{python}

There's a better way!
\end{subtopic}

\begin{subtopic}{6}
When working on large projects, you will want to be able to ensure you haven't broken anything when you make changes, this is where testing comes in handy!
\end{subtopic}

\begin{subtopic}{7}
This course has three assignments.

Each assignment will build on the previous assignment, working towards a large project.
\end{subtopic}

\end{topic}

\begin{topic}{Anatomy of a Java Program}
\begin{java}
public class HelloWorld {
    public static void main(String[] args) {
        System.out.println("Hello World");
    }
}
\end{java}

All java files need to have a class.

If we want to make a file called \texttt{HelloWorld.java} we would need a class named \texttt{HelloWorld} as demonstrated above.

The entry point for java programs is the main method, when a file is run the code in the main method is called.
\end{topic}

\begin{topic}{Tutorials}
\begin{itemize}
\item Tutorials are the way we teach most concepts that are tested in the exams (including writing code on paper).

\item Computers \textbf{should not} be used in tutorials.

\item This course has a very high failure rate - most people fail the exam so tutorials and lectures are important
\end{itemize}
\end{topic}

\begin{topic}{Questions}
\begin{subtopic}{1-4}
Implement \texttt{String num(int value)}: which for $0 < x < 10$ returns the name of the number, otherwise returns "??". This should be done twice, each time using one of the methods listed below:
\begin{itemize}
    \item Use a switch statement
    \item Use an array of Strings.
\end{itemize}
\\~\
\end{subtopic}

\begin{subtopic}{1,5,6}
The Fibbonacci sequence is defined as 1, 1, 2, . . . with F(n) = F(n - 1) + F(n - 2).
\begin{enumerate}
    \item Compute some more terms of the sequence to use for testing.
    \item Write \texttt{int fib(int n)}
    \item “Desk check” your Implementation.
\end{enumerate}
\end{subtopic}

\begin{subtopic}{2}
\begin{java}
public static String num(int value) {
    String result;
    switch (value) {
        case 1:
            result = "One";
            break;
        ...
        default:
            result = "??";
    }
    return result;
}
\end{java}
\end{subtopic}

\begin{subtopic}{3}
\begin{java}
public static String num(int value) {
    switch (value) {
        case 1:
            return "One";
        case 2:
            return "Two";
        ...
        default:
            return "??";
    }
}
\end{java}
\end{subtopic}

\begin{subtopic}{4}
\begin{java}
public static String num(int value) {
    String[] numbers = {"One", "Two", "Three", ..};
    if (value > 0 && value < 10) {
        return numbers[value + 1];
    }
    return "??";
}
\end{java}
\end{subtopic}

\begin{subtopic}{5}
1, 1, 2, 3, 5, 8, 13, 21, 34, 55, 89, 144
\end{subtopic}

\begin{subtopic}{6}
\begin{java}
public static int fib(int n) {
    if (n < 2) {
        return 1;
    }
    return fib(n - 1) + fib(n - 2);
}
\end{java}
\end{subtopic}

\end{topic}

\end{document} 
