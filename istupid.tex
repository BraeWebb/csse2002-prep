\documentclass[12]{beamer}
\usepackage{multicol}
\usepackage{enumitem}
\usepackage{listings}
\lstnewenvironment{python}{\lstset{language=python}}{}
\lstnewenvironment{java}{\lstset{style=customjava,language=java}}{}

\lstdefinestyle{customjava}{
  belowcaptionskip=1\baselineskip,
  breaklines=true,
  language=java,
  showstringspaces=false,
  basicstyle=\footnotesize\ttfamily,
  keywordstyle=\bfseries\color{green!40!black},
  commentstyle=\itshape\color{purple!40!black},
  identifierstyle=\color{blue},
  stringstyle=\color{orange},
  escapeinside={(*@}{@*)},
  tabsize=4
}

\usetheme[progressbar=frametitle]{metropolis}
\setbeamertemplate{frame numbering}[fraction]
\usefonttheme{metropolis}
\usecolortheme{spruce}
\setbeamercolor{background canvas}{bg=white}
\setbeamercovered{transparent=5}
\metroset{block=fill}

\definecolor{mygreen}{rgb}{125, 5, 25}
\usecolortheme[named=mygreen]{structure}



\title{Week 11 Tutorial}
\author{Brae}
\institute {CSSE2002: Programming in the Large}

\begin{document}

\begin{frame}
\titlepage
\end{frame}

\begin{frame}[t, fragile]{This Week} \vspace{4pt}
This week we will be working on questions from the \texttt{2017 Semester Two Final Exam Paper}.\\[20pt]
\begin{enumerate}
	\item[1.] Go to the UQ Library Website (library.uq.edu.au)
	\item[2.] Select \texttt{Past exam papers} from the search dropdown
	\item[3.] Enter \texttt{CSSE2002} into the search field
	\item[4.] Select \texttt{2017 Sem 2}
\end{enumerate}
\end{frame}

\begin{frame}[t, fragile]{Question One} \vspace{4pt}

The code below compiles and works correctly, but has several stylistic errors. Fix the code. Do not write comments.

\begin{onlyenv}<1>

\begin{java}
public class aclass {
public static int M(int N) {
	int x = 0;
		for(int i = 1;
		i<N;i=i+1)if
(N%i==0)x=x+1;return x;
}}
\end{java}

\end{onlyenv}

\begin{onlyenv}<2>

\textbf{Camel Case Naming}

\begin{java}
public class aClass {
public static int m(int n) {
	int x = 0;
		for(int i = 1;
		i<n;i=i+1)if
(n%i==0)x=x+1;return x;
}}
\end{java}

\end{onlyenv}

\begin{onlyenv}<3>

\textbf{Indentation}

\begin{java}
public class aClass {
	public static int m(int n) {
		int x = 0;
		for(int i = 1;i<n;i=i+1)
			if(n%i==0)x=x+1;return x;
}}
\end{java}

\end{onlyenv}

\begin{onlyenv}<4>

\textbf{Oneline Java Code}

\begin{java}
public class aClass {
	public static int m(int n) {
		int x = 0;
		for(int i = 1;i<n;i=i+1) {
			if(n%i==0) {
				x=x+1;
				return x;
			}
		}
	}
}
\end{java}

\end{onlyenv}

\begin{onlyenv}<5>

\textbf{Horizontal Whitespace}

\begin{java}
public class aClass {
	public static int m(int n) {
		int x = 0;
		for (int i = 1; i < n; i = i + 1) {
			if (n % i == 0) {
				x = x + 1;
				return x;
			}
		}
	}
}
\end{java}

\end{onlyenv}

\end{frame}

\begin{frame}[t, fragile]{Question Two} \vspace{4pt}

\begin{onlyenv}<1>
\begin{java}
public class Person {
	private String name;
	public Person(String n) { name = n; }
	public String getName() { return name; }
	public void setName(String n) { name = n ; }
}
\end{java}
The class invariant for this class is indented to be that only legal characters are used in a person’s name, as defined by the \texttt{legalNameCharacter} method below.

\begin{java}
public static boolean legalNameCharacter(char c) {
	return "a" <= c && c <= "z" k ||
		   "A" <= c && c <= "Z" k ||
		   c == "-" k || c == " ";
}
\end{java}

Add a JavaDoc comment which describes the invariant.
\end{onlyenv}

\begin{onlyenv}<2->
\begin{java}
/**
  * invariant: 
  * (*@
$
\forall x \in \mathbb{Z} | (\,x \leq 0 \land x < name.length)\, \Rightarrow legalNameCharacter(name[x])
$ @*)
  */
\end{java}
\end{onlyenv}

\begin{onlyenv}<3->
\begin{java}
/**
  * invariant: \forall char c; (name.contains(c)) 
  *                   ==> legalNameCharacter(c);
  */
\end{java}
\end{onlyenv}

\begin{onlyenv}<4->
\begin{java}
/**
  * invariant: for every character, c, in name, 
  *	       legalNameCharacter(c) returns true
  */
\end{java}
\end{onlyenv}

\end{frame}

\begin{frame}[t, fragile]{Question Three} \vspace{4pt}

\begin{java}
boolean isPrime(int n) { ... }

/** @require n > 0
  * @ensure \result is the number of primes between 2 and n inclusive
  */
int countPrimes(int n) {
	int count = 0;
	while (n >= 2) {
		if (isPrime(n))
			count = count + 1;
		n = n - 1;
	}
	return count;
}
\end{java}

\end{frame}

\begin{frame}[t, fragile]{Question Four} \vspace{4pt}

\begin{onlyenv}<1>
\begin{java}
public class A {
	public int m()
	public int m(int x, int y)
}

public class B extends A {
	public int m(double x, double y)
}

public class C extends B {
	public int m()
	public int m(int x, double y)
}
\end{java}

\begin{itemize}
	\begin{columns}[onlytextwidth]
	\column{0.3\textwidth}
	\item[1.] b.m();
	\item[2.] c.m(1, 2.1);

	\column{0.3\textwidth}
	\item[3.] b.m(1, 2.1);
	\item[4.] b.m(1, 2);

	\column{0.3\textwidth}
	\item[5.] c.m(1.1, 2.1);
	\item[6.] c.m(1, 2);
	\end{columns}
\end{itemize}

\end{onlyenv}

\begin{onlyenv}<2>
\begin{itemize}
	\item[1.] A.m()
	\item[2.] B.m(double, double)
	\item[3.] B.m(double, double)
	\item[4.] A.m(int, int)
	\item[5.] B.m(double, double)
	\item[6.] A.m(int, int)
\end{itemize}
\end{onlyenv}

\end{frame}

\end{document}
