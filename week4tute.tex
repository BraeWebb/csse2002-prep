\documentclass[week4]{csse2002}
\usepackage{slides}

\usepackage{tikz}
\usetikzlibrary{positioning}

\author{Brae Webb \& James Stuart}

\title{CSSE2002 Week 4 Tutorial}
 
\begin{document}

\begin{frame}
\maketitle
\end{frame}

\begin{topic}{Admin}
\begin{enumerate}
    \item Opportunity (SVN feedback)
    \item Assignment 1 Released
\end{enumerate}
\end{topic}

\begin{topic}{This Week...}
\begin{enumerate}
    \item Making Exceptions (hint hint, assignment)
    \item Throwing \& Catching Exceptions
    \item Overriding Object Methods
    \begin{enumerate}
    	\item equals
    	\item hashCode
    \end{enumerate}
    \item Implementing the Comparable Interface
\end{enumerate}
\end{topic}

\section{Exceptions}

\begin{topic}{Making Exceptions}
{\small Read: the easiest thing you'll ever do in this course}
\begin{subtopic}{1}
When something goes wrong, we throw an exception. Preferably a specific exception that somewhat explains what went wrong. 
\end{subtopic}
\begin{java}
public class CustomException extends Exception {}
\end{java}

\begin{subtopic}{2}
\begin{java}
public class HelloWorld {
	public static void main(String[] args) throws CustomException {
		throw new CustomException();
	}
}
\end{java}
\end{subtopic}

\begin{subtopic}{3-}
What if I want to add a message?
\end{subtopic}

\begin{subtopic}{4-}
\begin{java}
public class CustomException extends Exception {
	public CustomException() {
		super();
	}

	public CustomException(String message) {
		super(message);
	}
}
\end{java}
\end{subtopic}

\begin{subtopic}{5-}
\begin{java}
public class HelloWorld {
	public static void main(String[] args) throws CustomException {
		throw new CustomException("you done messed up a-a-ron");
	}
}
\end{java}
\end{subtopic}
\end{topic}

\begin{topic}{try, catch}
Whenever we want to perform some action that might go wrong (throw an exception), we want to \highlight{try} to perform these action(s), and \highlight{catch} any exception that gets thrown. 
\begin{subtopic}{1}
\begin{java}
try {
	// do some things
} catch (CustomException  e) {
	// We caught an exception!
}
\end{java}
\end{subtopic}

\begin{subtopic}{2}
\begin{java}
try {
	// do some things
} catch (CustomException  e) {
	// We caught an exception!
} catch (Exception e) {
	// We will catch all the (other) exceptions!
}
\end{java}
\end{subtopic}

\begin{itemize}
\item The \texttt{try} block stops executing when an exception is thrown. It doesn't continue. 
\item The \texttt{catch} block is like an \texttt{if} statement, checking if an exception thrown can be turned into a variable of type \highlight{CustomException} called \highlight{e}.
\begin{subtopic}{2}
\item  Multiple catch statements are checked in order. 
\item Remember implicit casting!
\end{subtopic}
\end{itemize}


\end{topic}

\begin{topic}{Runtime Exception?}
Runtime exceptions are used for errors that can be prevented
programmatically.

Examples: 
\begin{itemize}
	\item \textbf{NullPointerException}
	\item \textbf{ArrayIndexOutOfBoundException}
\end{itemize}

\begin{subtopic}{2-}
How do I make a runtime exception?
\end{subtopic}

\begin{subtopic}{3-}
\begin{java}
public class CustomRuntimeException extends RuntimeException {}
\end{java}
\end{subtopic}

\begin{subtopic}{4-}
What makes runtime exceptions special?
\end{subtopic}
\end{topic}

\begin{topic}{Finally, finally}
Finally is \highlight{always} executed. (exceptional cases: crashes, exiting the program or infinite loops)

\begin{java}
public int example() {
    try {
        return 20;
    } finally {
        System.out.println("always");
    }
}
\end{java}

Why is finally useful?
\end{topic}

\begin{topic}{Exception: Exercises}
	Let's have some practice!
\end{topic}

\section{Overriding Object Methods}
\begin{topic}{Overriding Object Methods}
The java \textbf{Object} class (which all classes extend) has a few
methods. Sometimes we find it useful to override these methods.

By default, java doesn't know much about our classes.

Today we'll be looking at:
\begin{itemize}
	\item boolean equals(Object object)
	\item int hashCode()
	\item int compareTo(Object object)
\end{itemize}
\end{topic}

\begin{topic}{.equals}
Java doesn't know how to compare two instances of a class.

\begin{subtopic}{1}
\begin{java}
public class Point {
	public int x;
	public int y;

	public Point(int x, int y) {
		this.x = x;
		this.y = y;
	}
}
\end{java}
\end{subtopic}

\begin{subtopic}{2}
\begin{java}
public class Point {
	public int x;
	public int y;

	public Point(int x, int y) {
		this.x = x;
		this.y = y;
	}

	public static void main(String[] args) {
		Point point1 = new Point(3, 10);
		Point point2 = new Point(3, 10);
		System.out.println(point1.equals(point2)); // false
	}
}
\end{java}
\end{subtopic}

\begin{subtopic}{3}
\begin{java}
public boolean equals(Object o) {
	if (!(o instanceof Point)) {
		return false;
	}
	Point other = (Point) o;

	return this.x == other.x && this.y == other.y;
}
\end{java}
\end{subtopic}
\end{topic}

\begin{topic}{.hashCode}
Only requirement for this course: if two objects are equal, their
hashcode has to be equal

\begin{subtopic}{2-}
\begin{java}
public int hashCode() {
    return this.x * this.y;
}
\end{java}
\end{subtopic}

\begin{subtopic}{3-}
\begin{java}
public int hashCode() {
    int hash = 7;
    hash = 31 * hash + this.x;
    hash = 31 * hash + this.y;
    return hash;
}
\end{java}
\end{subtopic}
\end{topic}

\begin{topic}{.compareTo}
If an object implements the Comparable interface, it needs to have a
compareTo method.

\begin{java}
public int compareTo(MyClass other)
\end{java}

If the current instance (\texttt{this}) is \highlight{equal} to the \texttt{other}, then this method returns \highlight{0}.

If the current instance is \highlight{greater} than \texttt{other}, it returns \highlight{positive}.

If the current instance is \highlight{lesser} than \texttt{other}, it returns \highlight{negative}.

What does it mean to be greater than or lesser than? Up to you!
\end{topic}

\end{document} 
